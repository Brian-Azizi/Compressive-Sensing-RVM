\chapter{Introduction}
The goal of the MPhil project is an implementation of a generic Compressive Sensing algorithm that can efficiently reconstruct video signals from a small number of measurements.

\emph{Compressive Sensing} (CS) \cite{candes2006, donoho2006} is a relatively recent framework within digital signal processing.
Using the techniques in CS, we can measure signals \emph{directly in a compressed format}, leading to highly efficient data acquisition protocols.
In application areas such as magnetic resonance imaging (MRI), which has an inherently slow data acquisition process, this could potentially result in large scan time reductions, thus helping patients and doctors.

There are two major components to any CS system.
Let $\bm v \in\mathbb{R}^M$ denote the digital signal of interest.
The first component is a \emph{sensing mechanism} that acquires the measurements
\begin{equation}
  \label{eqn:intro_cs}
  \bm\Theta\bm v = \bm y
\end{equation}
where $\bm\Theta$ is a known $N\times M$ matrix with $N<<M$.
The second component is a \emph{reconstruction algorithm} that recovers the original signal $\bm v$ from the CS measurements $\bm y$ by finding a solution to the under-determined system (\ref{eqn:intro_cs}).

In order to solve the system, we make the assumption that $\bm v$ has a sparse representation.
This means that we can perform a change-of-basis transformation
\begin{equation}
  \label{eqn:intro_basis}
  \bm v = \bm\Psi\bm w
\end{equation}
such that the transformed signal $\bm w$ is \emph{sparse}, i.e. it has only a small number of non-zero entries
This is a realistic assumption for a wide range of common signal types, such as digital images, audio, video and many kinds communication signals.

Letting $\bm\Phi=\bm\Theta\bm\Psi$, the system (\ref{eqn:intro_cs}) can then be expressed as
\begin{equation}
  \label{eqn:intro_cs2}
  \bm y = \bm\Phi\bm w.
\end{equation}

A variety of deterministic CS algorithms have been developed that attempt to find sparse solutions to (\ref{eqn:intro_cs2}).
For our implementation, however, we choose to approach the problem from the point of view of \emph{machine learning}.

In general, the aim of supervised machine learning is to learn a relationship $t = f(\bm x)$ between an input vector $\bm x$ and a target $t$.
If the target is a discrete variable the problem is known as \emph{classification}, whereas if $t$ is real-valued we refer to the problem as \emph{regression}.

To apply regression techniques to the CS problem (\ref{eqn:intro_cs2}), we regard the CS measurements $y^{(i)}$ as targets and treat the corresponding rows $\bm\varphi^{(i)}$ of the matrix $\bm\Phi$ is input vectors.
In this context, $\bm y$ is known as a \emph{target vector} and $\bm\Phi$ as the corresponding \emph{design matrix}.
A linear regression algorithm takes these quantities as its input and ``learns'' a set of \emph{weights} $\bm w$.
Given $\bm w$, the original signal can be computed using equation (\ref{eqn:intro_basis}).

For our code, we implemented the \emph{Relevance Vector Machine} (RVM) \cite{tipping2001,tipping2003} as it tends to give very sparse solutions $\bm w$.
This algorithm was used by \cite{pilikos2014} to successfully reconstruct highly undersampled image signals.
Comparing it to a range of classic deterministic CS reconstruction algorithms, \cite{pilikos2014} found that the RVM gave superior reconstruction performance.

In practical CS systems, the sensing mechanism $\bm\Theta$ is typically implemented within the actual sensing hardware.
To provide a range of reconstruction scenarios, our code simulates three types of sensing matrices: random Gaussian, random Bernoulli and random signal masks.

Of particular interest is the case when $\bm\Theta$ corresponds to a signal mask, so that $\bm y$ is a highly undersampled version of the signal $\bm v$.
The problem of solving (\ref{eqn:intro_cs}) is then equivalent to the problem \emph{signal interpolation}.
For this particular class of sensing mechanisms, \cite{pilikos2014} developed an extension to the RVM called \emph{Multi-Scale Cascade of Estimations} (MSCE).
The MSCE algorithm utilizes the \emph{multiresolution properties of Haar wavelets} to form a cascade of RVMs.
\cite{pilikos2014} investigated its performance in image interpolation and found that it can provide a significant boost in the reconstruction quality over a single RVM.
For our implementation, we extended the MSCE to the case of video interpolation.

\section*{Thesis Organization}
In this thesis, we discuss the underlying theory and implementation details for all the building blocks of our Compressive Sensing implementation.

We begin in Chapter \ref{ch:cs} with a discussion of the relevant background in digital signal processing.
A brief overview of the conventional approaches to signal acquisition and compression is provided and then contrasted with the novel Compressive Sensing framework.

In order to ensure a high quality reconstruction, it is important to find sparse representations of the signal.
Chapter \ref{ch:dwt} describes the Discrete Cosine Transform and the Discrete Wavelet Transform.
These two basis transforms are often used for their sparsifying properties, especially when applied to digital image and video signals.

In Chapter \ref{ch:rvm}, we derive the Sparse Bayesian Learning framework and discuss two different algorithms for training the RVM.

Chapter \ref{ch:msce} combines Compressive Sensing with Sparse Bayesian Learning to form the Bayesian Compressive Sensing framework.
We show how CS can be viewed from a Bayesian perspective and discuss the MSCE algorithm.

Our methods for handling video input are described in Chapter \ref{ch:video} and we provide further implementation details and practical considerations in Chapter \ref{ch:code}.

Chapter \ref{ch:results} contains a range of example results as well as a performance evaluation of the code.
Finally, we discuss future work and conclude in Chapter \ref{ch:conclusion}.

