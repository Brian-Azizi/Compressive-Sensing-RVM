\chapter{Introduction}

There are three parts: A signal processing framework called \emph{Compressive Sensing (CS)}, a pre-processing step in form of a basis transformation based on discrete wavelet transforms and a Machine Learning algorithm called \emph{Sparse Bayesian Learning}.

The key notion that ties in these three areas is the notion of \emph{sparsity}.

The $\ell_p$ -norm of a vector $\bm z \mathbb{R}^n$ is defined as follows for $p>0$:

\begin{equation}
\label{eqn:lp-norm}
  ||\bm z||_p \equiv \left( \sum_{i=1}^n |z_i|^p \right)^\frac{1}{p}.
\end{equation}
If $p=0$ we can define the $\ell_0$ ``norm'' of $\bm z$ to be the number of its non-zero entries:
\begin{equation}
\label{eqn:l0-norm}
  ||\bm z||_0 \equiv  \sum_{i=1}^n \mathbbm{1}\{z_j \neq 0\}
\end{equation}


Our contributions are three-fold:
\begin{enumerate}
\item An extension of the MSCE-algorithm in \cite{pilikos2014} to video data.
\item An extension to MSCE to deal with additional sensing mechanisms.
\item A parameter study and performance comparison of different wavelet representations, sensing modalities, 
\end{enumerate}


\section*{Background}
\section*{Thesis Organization}
