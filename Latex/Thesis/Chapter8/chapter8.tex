\chapter{Conclusion}
\label{ch:conclusion}

In this thesis, we investigated the efficacy of the Bayesian Compressive Sensing framework for efficient reconstruction of highly under-sampled video signals.
We developed an extension to the Multi-Scale Cascade of Estimations algorithm that achieves near-perfect reconstructions of videos from a very small set of measurements.

To do so, we constructed three-dimensional wavelet basis functions that allow for a highly compressible representation of the video signal.
Compressive Sensing inversion is then formulated as a machine learning problem and the Relevance Vector Machine was employed to find highly sparse solutions.
To boost performance, a cascade of RVMs it built that exploits the multi-resolution properties of wavelet basis functions.

In order to deal with the large memory requirements of the algorithm, the reconstruction is performed in blockwise fashion.
We have also implemented the method as a distributed program, resulting in dramatically reduced execution times.

Future research could improve performance by extending these methods in various ways.
Alternative sets of waveles, such as the CDF-9/7 wavelet that is used by the JPEG2000 format, may lead to sparser representations of the video signals. 
Furthermore, the speed of the algorithm can be increased by using a multi-threaded implementation of the Sequential Sparse Bayesian Learning Algorithm. 

Development of Bayesian approaches to Compressive Sensing systems is an active area of research.

