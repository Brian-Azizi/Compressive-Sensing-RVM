\chapter{Conclusion}
\label{ch:conclusion}

For the MPhil thesis, we developed a complete Compressive Sensing system that provides near-perfect reconstruction of video signals from as few as $0.3M$ compressed samples, where $M$ is the original signal length.

We constructed three-dimensional DCT and Haar wavelet basis functions that allow for a highly compressible representations of video signal.
The CS problem was then approached from a Bayesian machine learning perspective and we implemented the BCS algorithm as our main method for reconstruction.
In the case of undersampled video signals, we were able to get an additional boost in reconstruction quality by a using a cascade of RVMs that exploits the multi-resolution properties of wavelet functions.

In order to deal with the large data sets of the algorithm, our implementation can perform the reconstruction in a blockwise fashion.
Furthermore, we implemented the system as distributed program, so as to achieve practical runtimes.

Further research could improve the performance of the system.
In order to fully harness the power of the MSCE, more sophisticated wavelet functions should be used.
Alternative sets of wavelets, such as the CDF-9/7 wavelet that is used by the JPEG2000 format, can lead to sparser representations of the video signals. 
